\documentclass{beamer}
\usetheme[numbering=fraction]{metropolis}         % Use metropolis theme
\usepackage{xeCJK}

\title{计科专业工程教育认证}
\subtitle{师资队伍}
\date{\today}
\author{王以松、张海宁}


\institute{计科}
\begin{document}
\maketitle
\begin{frame}{目录}
\setbeamertemplate{section in toc}[sections numbered]
\tableofcontents[hideallsubsections]

\end{frame}
\section{认证标准}
\begin{frame}{认证标准}

\begin{enumerate}
\item
教师数量能满足教学需要,结构合理,并有企业或行业专家作为兼职教师。
\item
教师应具有足够的教学能力、专业水平、工程经验、沟通能力、职业发展能力,并且能够开展工程实践问题研究,参与学术交流。教师的工程背景应能满足专业教学的需要。
\item
教师应有足够时间和精力投入到本科教学和学生指导中,并积极参与教学研究与改革。
\item
教师应为学生提供指导、咨询、服务,并对学生职业生涯规划、职业从业教育有足够的指导。
\item
教师必须明确他们在教学质量提升过程中的责任,不断改进工作,满足培养目标要求。

\end{enumerate}
\end{frame}
\begin{frame}{补充标准}
\begin{enumerate}
\item
专业背景

大部分授课教师在其学习经历中至少有一个阶段是计算机专业的学历,部分教师具有相关专业学习的经历。
\item
工程背景
\begin{enumerate}
\item
通用
\begin{enumerate}
\item
授课教师应具备与自己所讲授的课程相匹配的计算机技术能力(包括操作能力、程序设计能力和解决问题的能力)、参加研究、工程设计实现。
\item
教师承担的课程数和授课学时数要限定在合理范围内,保证在教学以外有精力参加学术活动、工程和研究实践,以及提升个人的专业能力。
\end{enumerate}
\item
特殊
\begin{enumerate}
\item
讲授工程与应用类课程的教师应具有适当的工程背景。
\item
培养工程应用型人才为主的专业教师中,承担过工程性项目的教师需占有相当比例,就有教师具有与企业共同工作的经历。
\end{enumerate}
\end{enumerate}
\end{enumerate}
\end{frame}
\section{分解标准}
\begin{frame}{标准要求1及分解}
\begin{block}{标准1}

教师数量能满足教学需要,结构合理,并有企业或行业专家作为兼职教师。
\end{block}
\begin{block}{分解}
\begin{enumerate}
\item
数量、结构

拟从职称、学历、年龄、专业以及“各种人才称号”等这几个方面来描述。
\item
兼职教师

请学院领导聘请一定数量的企业或行业专家,来参与课程大纲的制定、指导毕业设计、参与毕业答辩、授课等环节。
\end{enumerate}
\end{block}
\end{frame}

\begin{frame}{标准要求2及分解}
\begin{block}{标准2}
教师应具有足够的教学能力、专业水平、工程经验、沟通能力、职业发展能力,并且能够开展工程实践问题研究,参与学术交流。教师的工程背景应能满足专业教学的需要。
\end{block}
\begin{block}{分解}
\begin{enumerate}
\item
专业背景及经历
\item
科研项目、论文、专利、咨询活动、学术交流活动
\item
教学水平评估
\end{enumerate}
\end{block}
\end{frame}

\begin{frame}{标准要求3及分解}
\begin{block}{标准3}
教师应有足够时间和精力投入到本科教学和学生指导中,并积极参与教学研究与改革。
\end{block}
\begin{block}{分解}
\begin{enumerate}
\item
教学工作量统计

\item
教改项目、精品课程
\end{enumerate}
\end{block}
\end{frame}

\begin{frame}{标准要求4及分解}
\begin{block}{标准4}
教师应为学生提供指导、咨询、服务,并对学生职业生涯规划、职业从业教育有足够的指导。
\end{block}
\begin{block}{分解}
\begin{enumerate}
\item
专业宣讲、专业班主任、学术讲座
\item
SRT、学科竞赛等

\item
学业导师
\item
各个实验室团队培养本科生
\item
指导大创

\end{enumerate}
\end{block}
\end{frame}

\begin{frame}{标准要求5及分解}
\begin{block}{标准5}
教师必须明确他们在教学质量提升过程中的责任,不断改进工作,满足培养目标要求。
\end{block}
\begin{block}{分解}
\begin{enumerate}
\item
制定课程教学大纲的过程就是明确责任的过程

\item
学院督导团
\item
参与教改

\end{enumerate}
\end{block}
\end{frame}

\begin{frame}{补充标准要求1及分解}
\begin{block}{补充标准要求1}
专业背景

大部分授课教师在其学习经历中至少有一个阶段是计算机专业的学历,部分教师具有相关专业学习的经历。
\end{block}
\begin{block}{分解}

教师专业背景

\end{block}
\end{frame}

\begin{frame}{补充标准要求2及分解}
\begin{block}{补充标准要求2}
工程背景

授课教师应具备与自己所讲授的课程相匹配的\textcolor{red}{计算机技术能力}(包括操作能力、程序设计能力和解决问题的能力)、参加研究、工程设计实现。
教师承担的课程数和授课学时数要\textcolor{red}{限定在合理范围内},保证在教学以外有精力参加学术活动、工程和研究实践,以及提升个人的专业能力。
讲授工程与应用类课程的教师应具有适当的\textcolor{red}{工程背景}。
培养工程应用型人才为主的专业教师中,\textcolor{red}{承担过工程性项}目的教师需占有相当比例,应有教师具有\textcolor{red}{与企业共同工作}的经历。

\end{block}
\begin{block}{分解}
与工程有关的证书、工程项目、公司工作经历、教学工作量


\end{block}
\end{frame}
\section{工作计划}
{%
\setbeamertemplate{frame footer}{7月之前要完成}
\begin{frame}{工作计划}
\begin{enumerate}
\item
参考机械学院的材料,制定本专业教师相关材料的收集计划
\item
思考本专业教师方面的亮点
\item
\alert{调研学习外校的师资情况}
\end{enumerate}

\end{frame}
}
\begin{frame}[standout]
The END.

Q\&A
\end{frame}
\end{document}