% !TEX encoding = UTF-8 Unicode
%%%%%%%%%%%%%%%%%%%%%%%%%%%%%%%%%%%%%%%%%
% Beamer Presentation
% LaTeX Template
% Version 1.0 (10/11/12)
%
% This template has been downloaded from:
% http://www.LaTeXTemplates.com
%
% License:
% CC BY-NC-SA 3.0 (http://creativecommons.org/licenses/by-nc-sa/3.0/)
%
%%%%%%%%%%%%%%%%%%%%%%%%%%%%%%%%%%%%%%%%%

%----------------------------------------------------------------------------------------
%	PACKAGES AND THEMES
%----------------------------------------------------------------------------------------

\documentclass{beamer}

\mode<presentation> {

% The Beamer class comes with a number of default slide themes
% which change the colors and layouts of slides. Below this is a list
% of all the themes, uncomment each in turn to see what they look like.

%\usetheme{default}
%\usetheme{AnnArbor}
%\usetheme{Antibes}
%\usetheme{Bergen}
%\usetheme{Berkeley}
%\usetheme{Berlin}
%\usetheme{Boadilla}
%\usetheme{CambridgeUS}
%\usetheme{Copenhagen}
%\usetheme{Darmstadt}
%\usetheme{Dresden}
%\usetheme{Frankfurt}
%\usetheme{Goettingen}
%\usetheme{Hannover}
%\usetheme{Ilmenau}
%\usetheme{JuanLesPins}
%\usetheme{Luebeck}
%\usetheme{Madrid}
%\usetheme{Malmoe}
%\usetheme{Marburg}
%\usetheme{Montpellier}
%\usetheme{PaloAlto}
%\usetheme{Pittsburgh}
%\usetheme{Rochester}
\usetheme{Singapore}
%\usetheme{Szeged}
%\usetheme{Warsaw}

% As well as themes, the Beamer class has a number of color themes
% for any slide theme. Uncomment each of these in turn to see how it
% changes the colors of your current slide theme.

%\usecolortheme{albatross}
%\usecolortheme{beaver}
%\usecolortheme{beetle}
%\usecolortheme{crane}
%\usecolortheme{dolphin}
%\usecolortheme{dove}
%\usecolortheme{fly}
%\usecolortheme{lily}
%\usecolortheme{orchid}
%\usecolortheme{rose}
%\usecolortheme{seagull}
%\usecolortheme{seahorse}
%\usecolortheme{whale}
%\usecolortheme{wolverine}

%\setbeamertemplate{footline} % To remove the footer line in all slides uncomment this line
%\setbeamertemplate{footline}[page number] % To replace the footer line in all slides with a simple slide count uncomment this line

%\setbeamertemplate{navigation symbols}{} % To remove the navigation symbols from the bottom of all slides uncomment this line
}

\usepackage{graphicx} % Allows including images
\usepackage{booktabs} % Allows the use of \toprule, \midrule and \bottomrule in tables
\usepackage{xeCJK}
\setCJKmainfont{SourceHanSerif-Regular}
\usepackage{color}
\usepackage{listings}
\lstset{numbers=left}
\usepackage{fancyvrb}%use Verbatim-the extended verbatim
\usepackage{tikz}
\usepackage{amsmath}

%----------------------------------------------------------------------------------------
%	TITLE PAGE
%----------------------------------------------------------------------------------------

\title[Django]{python} % The short title appears at the bottom of every slide, the full title is only on the title page
\subtitle{data process}
\author{张海宁} % Your name
\institute[计算机科学与技术学院] % Your institution as it will appear on the bottom of every slide, may be shorthand to save space
{
贵州大学 \\ % Your institution for the title page
\medskip
\textit{hnzhang1@gzu.edu.cn} % Your email address
}
\date{\today} % Date, can be changed to a custom date

\begin{document}

\begin{frame}
\titlepage % Print the title page as the first slide
\end{frame}
\begin{frame}{Overview}
\tableofcontents
\end{frame}
%\begin{Verbatim}[numbers=left,frame=single,rulecolor=\color{red}]

%\end{Verbatim}



%\begin{columns}
%\column{2.8cm}{if}
%\column{2.8cm}{if-else}
%\column{3.1cm}{if-elif}
%\begin{Verbatim}[numbers=none,frame=single,rulecolor=\color{red}]
%\column{3.1cm}{if-elif-else}
%\end{columns}
\section{dictionary、set}
\begin{frame}{映射}
两个非空集合$A$与$B$间存在着对应关系$f$,而且对于$A$中的每一个元素$x$,$B$中总有有唯一的一个元素$y$与它对应,就这种对应为从$A$到$B$的\textcolor{red}{映射},记作$f: A→B$。其中,$b$称为元素$a$在映射f下的象,记作:$b=f(a)$。$a$称为$b$关于映射f的原象。集合$A$中所有元素的象的集合称为映射f的值域,记作$f(A)$。
\end{frame}
\begin{frame}[fragile]{字典}
$Python$中用来处理\textcolor{red}{映射关系}的数据结构叫做\textcolor{red}{字典}$(dictionary)$。
\begin{itemize}
\item
在字典中,涉及两个集合,一个是\textcolor{red}{键}$(key)$的集合,一个是\textcolor{red}{值}$(value)$的集合。
\item
在字典中,元素是以“键:值”对的方式存储的。
\end{itemize}
\begin{block}{定义一个包含3个元素的字典}
\begin{Verbatim}[numbers=left,frame=single,rulecolor=\color{red}]
pac = {'0851':'贵阳', '023':'重庆', '028':'成都'}
\end{Verbatim}
\end{block}
\begin{block}{注意事项}
\begin{itemize}
\item 键必须是不可变对象
\item 值可以是任何类型的数据
\item 键是唯一的,值不必唯一
\item 字典中的元素是无序的
\end{itemize}

\end{block}
\end{frame}
\begin{frame}
\tiny{
\begin{table}[htp]
\caption{字典常用操作}
\begin{center}
\begin{tabular}{cp{11em}cp{11em}}
\toprule
\textbf{操作}&\textbf{描述}&\textbf{操作}&\textbf{描述}\\
\midrule
len(pac)&求元素个数&pac[key]&返回key对应的值\\
x in pac&x是否是pac的一个键&x:y not in pac&x:y这个元素是否不在pac中\\
pac[k1]=v1&if k1 in pac: 更新k1对应的值;否则将k1:v1这个元素添加到pac中&del pac[k1]&删除键为k1的元素\\
pac[k1,dft]&if k1 in pac: return pac[k1] else: return dft&d.clear()&移除字典中所有的元素\\
list(pac.keys())&返回pac中的所有键组成的列表&list(pac.values())&返回pac中的所有值组成的列表\\
list(pac.items())&返回(key,value)形式的二元组组成的列表&pac.update(c)&将字典c中所有元素合并入pac,若拥有相同的键,则使用c中的值替换pac中的值\\
\bottomrule
\end{tabular}
\end{center}
\label{dict}
\end{table}%
}
\end{frame}
\begin{frame}[fragile]{集合}
集合也是$python$中用来存储一系列元素的一种数据结构,其有以下特点:
\begin{itemize}
\item 其中的元素是无序的
\item 不允许重复元素
\item 可以包含数值、字符串、元组和布尔变量,不可以容纳列表或集合
\end{itemize}
\begin{block}{定义一个包含6个元素的集合}
\begin{Verbatim}[numbers=left,frame=single,rulecolor=\color{red}]
ns = {1,2,3,'贵阳','重庆',True}
\end{Verbatim}
\end{block}
\end{frame}

\begin{frame}
%\tiny{
\begin{table}[htp]
\caption{集合常用操作}
\begin{center}
\begin{tabular}{cc}
\toprule
\textbf{操作}&\textbf{描述}\\
\midrule
len(ns)&求集合ns中的元素个数\\
ns.add(e)&向集合ns中添加元素e\\
ns.clear()&移除集合ns中的所有元素\\
e in ns&元素e是否在集合ns中\\
set1.union(set2)&$set1\cup set2=\{x|x\in set1,or\ x\in set2\}$\\
set1.intersection(set2)&$set1\cap set2=\{x|x\in set1,and\  x\in set2\}$\\
set1.difference(set2)&$set1-set2=\{x|x\in set1,and\  x\notin set2\}$\\
\bottomrule
\end{tabular}
\end{center}
\label{dict}
\end{table}%
%}
\end{frame}
\begin{frame}[fragile]{遍历字典和集合中的元素}
\begin{Verbatim}[numbers=left,frame=single,rulecolor=\color{red}]
pac={'0851':'贵阳','023':'重庆','028':'成都'}
for i in pac.keys():
    print(i,"::",pac[i])
    
ns={1,2,3,'r'}
for i in ns:
    print(i)

1
2
3
r
0851 :: 贵阳
023 :: 重庆
028 :: 成都
\end{Verbatim}
\end{frame}

\section{read and write file}
\begin{frame}{文件}
截至目前,程序的运行结果显示在屏幕上,并最终随着关闭程序或系统,结果会消失。在实际应用中,通常希望能够\textcolor{red}{将程序运行结果长久地保存在磁盘的文件中},以供以后使用。
\end{frame}
\begin{frame}[fragile]{open函数}
$open()$函数可以用来打开或创建一个文件。
\small{
\begin{Verbatim}[numbers=left,frame=single,rulecolor=\color{red}]
open(file, mode='r', encoding=None)
===== ============================================
Character Meaning
----- --------------------------------------------
'r'   open for reading (default)
'w'   open for writing, truncating the file first
'x'   create a new file and open it for writing
'a'   open for writing, appending to the end of the file
       if it exists
'b'   binary mode
't'   text mode (default)
===== ==========================================
\end{Verbatim}
}
\end{frame}

\begin{frame}[fragile]{读取文件}

\begin{columns}
\column{.7\textwidth}
\begin{Verbatim}[numbers=left,frame=single,rulecolor=\color{red}]
f = open('./toolbox.py')
line_number = 1
for line in f:
    print(line_number, ':', line)
    line_number += 1
f.close()
\end{Verbatim}
\column{.3\textwidth}
\tiny{
\begin{Verbatim}[numbers=none,frame=none,rulecolor=\color{red}]
line1 使用open()函数打开文本文件

line3 使用for循环遍历文件的每一行

  
line6 使用close()函数关闭文件
\end{Verbatim}
}
\end{columns}
\end{frame}

\begin{frame}[fragile]{写文件}

\begin{columns}
\column{.7\textwidth}
\begin{Verbatim}[numbers=left,frame=single,rulecolor=\color{red}]
out = open('./open4Writing.py','w')
out.write("test line 1\n")
out.close()
out1 = open('./open4Writing.py','w')
out1.write("test line 2\n")
out.close()

out2 = open('appendFile.txt','a')
out2.write('first\n')
out2.close()
out3 = open('appendFile.txt','a')
out3.write('second\n')
out3.close()
\end{Verbatim}
\column{.3\textwidth}
\tiny{
\begin{Verbatim}[numbers=none,frame=none,rulecolor=\color{red}]
line1 使用open()函数打开文本文件
注意使用'w'模式
即open a file for writing

line234 使用write()函数写入字符串
注意\n的使用,是为了换行
  
line5 使用close()函数关闭文件

在电脑上进行写文件的测试:
比较'w'和'a'两种模式有何异同。
\end{Verbatim}
}
\end{columns}
\end{frame}
\section{Regular expression operations}

\begin{frame}{正则表达式}
\begin{block}{概念}
正则表达式是一个\textcolor{red}{特殊的字符序列},是一个\textcolor{red}{模式字符串},可以检测一个字符串是否与这个特殊的字符序列的\textcolor{red}{模式}相\textcolor{red}{匹配}。
\end{block}
\begin{block}{$re$}
$Python$中,$re$模块提供了正则表达式的功能。
\end{block}
\end{frame}
\begin{frame}[fragile]{模式字符串\footnote{正则表达式}}
\small{
\begin{columns}
\column{.5\textwidth}
\begin{Verbatim}[numbers=none,frame=none,rulecolor=\color{red}]
h[ea]llo     匹配hello或hallo
[a-z]        匹配任一小写字母
[A-Z]        匹配任一大写字母
[0-9]        匹配任一数字
[a-z0-9A-Z]  匹配任一数字或字母
[^abc]       匹配abc以外的字符
[^0-9]       匹配数字以外的字符
\end{Verbatim}

\column{.5\textwidth}
\begin{Verbatim}[numbers=none,frame=none,rulecolor=\color{red}]
\W  匹配任一非单词字符
\s  匹配任一空白字符
\S  匹配任一非空白字符
\d  匹配任一数字
\w  匹配任一数字、字母、下划线
.   匹配\n以外的字符
\D  匹配数字以外的字符
\end{Verbatim}
\end{columns}
}
\end{frame}

\begin{frame}[fragile]{模式字符串\footnote{正则表达式}}
\small{
\begin{columns}
\column{.5\textwidth}
\begin{table}
\begin{tabular}{cccc}
\toprule
\textbf{}&\textbf{}&\textbf{}&\textbf{}\\
\midrule
^&&$&\\
\bottomrule
\end{tabular}
\end{table}


\column{.5\textwidth}
\begin{Verbatim}[numbers=none,frame=none,rulecolor=\color{red}]
\W  匹配任一非单词字符
\s  匹配任一空白字符
\S  匹配任一非空白字符
\d  匹配任一数字
\w  匹配任一数字、字母、下划线
.   匹配\n以外的字符
\D  匹配数字以外的字符
\end{Verbatim}
\end{columns}
}
\end{frame}

\begin{frame}{匹配}
\end{frame}
%----------------------------------------------------------------------------------------
%	PRESENTATION SLIDES
%----------------------------------------------------------------------------------------
\section{homework}
\begin{frame}[fragile]{Homework}

\begin{enumerate}
\item
统计一个给定文档$aboutUN.txt$的词频,并将结果写入到$wordsFrequence.txt$中。\footnote{不统计标点符号,不考虑有单词换行的情况。}样式如下所示:
\begin{Verbatim}[numbers=left,frame=single,rulecolor=\color{red}]
UN    5
China    6
…
\end{Verbatim}
\item
通过$UN.txt$文件,获取目前联合国的193个会员国名字,并将其写入$UNmembers.txt$中。样式如下所示:
\begin{Verbatim}[numbers=left,frame=single,rulecolor=\color{red}]
China
America
…
\end{Verbatim}
%\item
%找零钱。输入收费金额和客户所付的钱,给出需要怎么找零。要求找零的钱币张数最少。(即需要多少1元的,多少2元的,多少5元的等)
\end{enumerate}
\end{frame}
\section{Q\&A}
\begin{frame}
\center{\Huge{Q\&A}}
\end{frame}


%How do I uninstall?
%
%	1.	Remove /Applications/Wireshark.app
%	2.	Remove /Library/Application Support/Wireshark
%	3.	Remove the wrapper scripts from /usr/local/bin
%	4.	Unload the org.wireshark.ChmodBPF.plist launchd job
%	5.	Remove /Library/LaunchDaemons/org.wireshark.ChmodBPF.plist
%	6.	Remove the access_bpf group.
%	7.	Remove /etc/paths.d/Wireshark
%	8.	Remove /etc/manpaths.d/Wireshark
\end{document} 

