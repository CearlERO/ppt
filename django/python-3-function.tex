% !TEX encoding = UTF-8 Unicode
%%%%%%%%%%%%%%%%%%%%%%%%%%%%%%%%%%%%%%%%%
% Beamer Presentation
% LaTeX Template
% Version 1.0 (10/11/12)
%
% This template has been downloaded from:
% http://www.LaTeXTemplates.com
%
% License:
% CC BY-NC-SA 3.0 (http://creativecommons.org/licenses/by-nc-sa/3.0/)
%
%%%%%%%%%%%%%%%%%%%%%%%%%%%%%%%%%%%%%%%%%

%----------------------------------------------------------------------------------------
%	PACKAGES AND THEMES
%----------------------------------------------------------------------------------------

\documentclass{beamer}

\mode<presentation> {

% The Beamer class comes with a number of default slide themes
% which change the colors and layouts of slides. Below this is a list
% of all the themes, uncomment each in turn to see what they look like.

%\usetheme{default}
%\usetheme{AnnArbor}
%\usetheme{Antibes}
%\usetheme{Bergen}
%\usetheme{Berkeley}
%\usetheme{Berlin}
%\usetheme{Boadilla}
%\usetheme{CambridgeUS}
%\usetheme{Copenhagen}
%\usetheme{Darmstadt}
%\usetheme{Dresden}
%\usetheme{Frankfurt}
%\usetheme{Goettingen}
%\usetheme{Hannover}
%\usetheme{Ilmenau}
%\usetheme{JuanLesPins}
%\usetheme{Luebeck}
%\usetheme{Madrid}
%\usetheme{Malmoe}
%\usetheme{Marburg}
%\usetheme{Montpellier}
%\usetheme{PaloAlto}
%\usetheme{Pittsburgh}
%\usetheme{Rochester}
\usetheme{Singapore}
%\usetheme{Szeged}
%\usetheme{Warsaw}

% As well as themes, the Beamer class has a number of color themes
% for any slide theme. Uncomment each of these in turn to see how it
% changes the colors of your current slide theme.

%\usecolortheme{albatross}
%\usecolortheme{beaver}
%\usecolortheme{beetle}
%\usecolortheme{crane}
%\usecolortheme{dolphin}
%\usecolortheme{dove}
%\usecolortheme{fly}
%\usecolortheme{lily}
%\usecolortheme{orchid}
%\usecolortheme{rose}
%\usecolortheme{seagull}
%\usecolortheme{seahorse}
%\usecolortheme{whale}
%\usecolortheme{wolverine}

%\setbeamertemplate{footline} % To remove the footer line in all slides uncomment this line
%\setbeamertemplate{footline}[page number] % To replace the footer line in all slides with a simple slide count uncomment this line

%\setbeamertemplate{navigation symbols}{} % To remove the navigation symbols from the bottom of all slides uncomment this line
}

\usepackage{graphicx} % Allows including images
\usepackage{booktabs} % Allows the use of \toprule, \midrule and \bottomrule in tables
\usepackage{xeCJK}
\setCJKmainfont{SourceHanSerif-Regular}
\usepackage{color}
\usepackage{listings}
\lstset{numbers=left}
\usepackage{fancyvrb}%use Verbatim-the extended verbatim
\usepackage{tikz}


%----------------------------------------------------------------------------------------
%	TITLE PAGE
%----------------------------------------------------------------------------------------

\title[Django]{python} % The short title appears at the bottom of every slide, the full title is only on the title page
\subtitle{function}
\author{} % Your name
\institute[计算机科学与技术学院] % Your institution as it will appear on the bottom of every slide, may be shorthand to save space
{
贵州大学 \\ % Your institution for the title page
\medskip
\textit{hnzhang1@gzu.edu.cn} % Your email address
}
\date{\today} % Date, can be changed to a custom date

\begin{document}

\begin{frame}
\titlepage % Print the title page as the first slide
\end{frame}
\begin{frame}{Overview}
\tableofcontents
\end{frame}
%\begin{Verbatim}[numbers=left,frame=single,rulecolor=\color{red}]

%\end{Verbatim}



%\begin{columns}
%\column{2.8cm}{if}
%\column{2.8cm}{if-else}
%\column{3.1cm}{if-elif}
%\begin{Verbatim}[numbers=none,frame=single,rulecolor=\color{red}]
%\column{3.1cm}{if-elif-else}
%\end{columns}
\section{introduce of function}
\begin{frame}{introduce of function}
函数(方法)具有以下特点:
\begin{itemize}
\item 是一段代码
\item 能完成特定功能
\item 可以在其他地方被调用
\item 可以接收参数或不接收参数
\item 可以有返回值或没有返回值
\end{itemize}
函数用于将一个\textcolor{red}{复杂}的问题\textcolor{red}{分解}为若干个\textcolor{red}{简单}的子问题。
\end{frame}
\begin{frame}{函数的类别}
\begin{description}
\item[ 内建函数 ]  int()、chr()、ord()、round()等
\item[ 自定义函数 ] 
\end{description}

\end{frame}
\section{define a function}
\begin{frame}[fragile]{How to define a function}
\begin{Verbatim}[numbers=left,frame=single,rulecolor=\color{red}]
def functionName(parameter1, parameter2, …):
    functionBlock
    return aValue
\end{Verbatim}
\begin{description}
\item[ def ] 定义函数的关键字
\item[ functionName ] 函数名
\item[ parameter1…] 参数\footnote{函数可以不需要参数}
\item[ functionBlock ] 函数体
\item[ return ] 表明此函数是有返回值的\footnote{函数可以不返回任何值}
\item[ aValue ] 函数的返回值 
\end{description}

\end{frame}

\begin{frame}[fragile]{Function example - return nothing}
\begin{Verbatim}[numbers=left,frame=single,rulecolor=\color{red}]
def myGCD(a,b):
    c,d=a,b
    while a%b !=0:
        a,b = b,a%b
    print(c,'和',d,'的最大公约数是:',b)

a = int(input("a:"))
b = int(input("b:"))
myGCD(a,b)
print(math.gcd(a,b))
\end{Verbatim}
\end{frame}

\begin{frame}[fragile]{Function example - return nothing and have no parametes}
\begin{Verbatim}[numbers=left,frame=single,rulecolor=\color{red}]
def print_squre():
    print(' ----')
    print('|    |')
    print('|    |')
    print(' ----')

print_squre()
\end{Verbatim}
\end{frame}

\begin{frame}[fragile]{Function example - return a list}
\begin{Verbatim}[numbers=left,frame=single,rulecolor=\color{red}]
def is_prime(first_n):
    prime_list = [2]
    i = 3
    while True:
        for d in range(2, i):
            if i % d == 0:
                break
            elif (i % d != 0) and (d == (i-1)):
                prime_list.append(i)
        if len(prime_list) == first_n:
            break
        i += 1
    return prime_list

n = int(input("n:"))
print(is_prime(n))
\end{Verbatim}
\end{frame}
%----------------------------------------------------------------------------------------
%	PRESENTATION SLIDES
%----------------------------------------------------------------------------------------
\section{homework}
\begin{frame}{Homework}
\begin{enumerate}
\item
求出前10个素数\footnote{质数(prime number)又称素数,有无限个。质数定义为在大于1的自然数中,除了1和它本身以外不再有其他因数。}。
\item
使用辗转相除法\footnote{用较大数除以较小数,再用出现的余数(第一余数)去除除数,再用出现的余数(第二余数)去除第一余数,如此反复,直到最后余数是0为止。如果是求两个数的最大公约数,那么最后的除数就是这两个数的最大公约数。}求任意两个数的最大公约数。
%\item
%找零钱。输入收费金额和客户所付的钱,给出需要怎么找零。要求找零的钱币张数最少。(即需要多少1元的,多少2元的,多少5元的等)
\end{enumerate}
\end{frame}
\section{Q\&A}
\begin{frame}
\center{\Huge{Q\&A}}
\end{frame}


%How do I uninstall?
%
%	1.	Remove /Applications/Wireshark.app
%	2.	Remove /Library/Application Support/Wireshark
%	3.	Remove the wrapper scripts from /usr/local/bin
%	4.	Unload the org.wireshark.ChmodBPF.plist launchd job
%	5.	Remove /Library/LaunchDaemons/org.wireshark.ChmodBPF.plist
%	6.	Remove the access_bpf group.
%	7.	Remove /etc/paths.d/Wireshark
%	8.	Remove /etc/manpaths.d/Wireshark
\end{document} 

