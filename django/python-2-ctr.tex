% !TEX encoding = UTF-8 Unicode
%%%%%%%%%%%%%%%%%%%%%%%%%%%%%%%%%%%%%%%%%
% Beamer Presentation
% LaTeX Template
% Version 1.0 (10/11/12)
%
% This template has been downloaded from:
% http://www.LaTeXTemplates.com
%
% License:
% CC BY-NC-SA 3.0 (http://creativecommons.org/licenses/by-nc-sa/3.0/)
%
%%%%%%%%%%%%%%%%%%%%%%%%%%%%%%%%%%%%%%%%%

%----------------------------------------------------------------------------------------
%	PACKAGES AND THEMES
%----------------------------------------------------------------------------------------

\documentclass{beamer}

\mode<presentation> {

% The Beamer class comes with a number of default slide themes
% which change the colors and layouts of slides. Below this is a list
% of all the themes, uncomment each in turn to see what they look like.

%\usetheme{default}
%\usetheme{AnnArbor}
%\usetheme{Antibes}
%\usetheme{Bergen}
%\usetheme{Berkeley}
%\usetheme{Berlin}
%\usetheme{Boadilla}
%\usetheme{CambridgeUS}
%\usetheme{Copenhagen}
%\usetheme{Darmstadt}
%\usetheme{Dresden}
%\usetheme{Frankfurt}
%\usetheme{Goettingen}
%\usetheme{Hannover}
%\usetheme{Ilmenau}
%\usetheme{JuanLesPins}
%\usetheme{Luebeck}
%\usetheme{Madrid}
%\usetheme{Malmoe}
%\usetheme{Marburg}
%\usetheme{Montpellier}
%\usetheme{PaloAlto}
%\usetheme{Pittsburgh}
%\usetheme{Rochester}
\usetheme{Singapore}
%\usetheme{Szeged}
%\usetheme{Warsaw}

% As well as themes, the Beamer class has a number of color themes
% for any slide theme. Uncomment each of these in turn to see how it
% changes the colors of your current slide theme.

%\usecolortheme{albatross}
%\usecolortheme{beaver}
%\usecolortheme{beetle}
%\usecolortheme{crane}
%\usecolortheme{dolphin}
%\usecolortheme{dove}
%\usecolortheme{fly}
%\usecolortheme{lily}
%\usecolortheme{orchid}
%\usecolortheme{rose}
%\usecolortheme{seagull}
%\usecolortheme{seahorse}
%\usecolortheme{whale}
%\usecolortheme{wolverine}

%\setbeamertemplate{footline} % To remove the footer line in all slides uncomment this line
%\setbeamertemplate{footline}[page number] % To replace the footer line in all slides with a simple slide count uncomment this line

%\setbeamertemplate{navigation symbols}{} % To remove the navigation symbols from the bottom of all slides uncomment this line
}

\usepackage{graphicx} % Allows including images
\usepackage{booktabs} % Allows the use of \toprule, \midrule and \bottomrule in tables
\usepackage{xeCJK}
\setCJKmainfont{SourceHanSerif-Regular}
\usepackage{color}
\usepackage{listings}
\lstset{numbers=left}
\usepackage{fancyvrb}%use Verbatim-the extended verbatim
\usepackage{tikz}


%----------------------------------------------------------------------------------------
%	TITLE PAGE
%----------------------------------------------------------------------------------------

\title[Django]{python} % The short title appears at the bottom of every slide, the full title is only on the title page
\subtitle{控制结构}
\author{} % Your name
\institute[计算机科学与技术学院] % Your institution as it will appear on the bottom of every slide, may be shorthand to save space
{
贵州大学 \\ % Your institution for the title page
\medskip
\textit{hnzhang1@gzu.edu.cn} % Your email address
}
\date{\today} % Date, can be changed to a custom date

\begin{document}

\begin{frame}
\titlepage % Print the title page as the first slide
\end{frame}
\begin{frame}{Overview}
\tableofcontents
\end{frame}
\section{关系和逻辑运算符}
\begin{frame}{条件}
控制结构包括:判断和循环。为了作出判断或控制循环,必须指定条件来做依据。

条件(也叫布尔表达式)是一种包含了关系运算符($>$、$>=$、$==$、in、not in等)和(或)逻辑运算符($and$、$or$、$not$)的表达式。

条件的结果为:True或False。
\end{frame}
\begin{frame}[fragile]{ASCII}
在应用关系运算符进行比较字符时的依据是这些字符的ASCII\footnote{ASCII(American Standard Code for Information Interchange,美国信息交换标准代码)是基于拉丁字母的一套电脑编码系统。}值。
\begin{block}{}
\begin{Verbatim}[numbers=left,frame=single,rulecolor=\color{red}]
ord(c, /)
    Return the ASCII value for a one-character string.
chr(i, /)
    Return a  string of the ASCII, …
>>> print("ord('A')",ord('A'),"; ord('Z')",ord('Z'),\
... "| ord('a')",ord('a'),"; ord('z')",ord('z'))
ord('A') 65 ; ord('Z') 90 | ord('a') 97 ; ord('z') 122
>>> print("chr(48)",chr(48),"; chr(57)",chr(57),\
... "| chr(169)",chr(169),"; chr(248)",chr(248))
chr(48) 0 ; chr(57) 9 | chr(169) © ; chr(248) ø
\end{Verbatim}
\end{block}

\end{frame}

\begin{frame}[fragile]{关系运算符小练习}
判断下列条件的结果。
\begin{Verbatim}[numbers=left,frame=single,rulecolor=\color{red}]
1<=1
'car'<'cat'
'Dog'<'dog'
'fun' in 'refunded'
'B' not in ('a','b','c')
let a=4,b=3,c='hello':
    (len(c)-a)==(b/2)
\end{Verbatim}
\end{frame}

\begin{frame}[fragile]{逻辑运算符}
当需要更复杂的条件时,就需要使用逻辑运算符,使用了这些运算符的条件称为\textcolor{red}{复合条件}。
\begin{block}{判断方法}
\begin{description}
\item[ a \textcolor{red}{$and$} b ]  a, b均为真时,结果为真
\item[ a \textcolor{red}{$or$} b ]  a, b至少一个为真时,结果为真
\item[ \textcolor{red}{$not$} a ]  a为假时,结果为真
\end{description}
\end{block}
\begin{block}{短路求值}
\begin{description}
\item[ a \textcolor{red}{$and$} b ]  当a为假时,还有必要判断b吗?
\item[ a \textcolor{red}{$or$} b ]  当a为真时,还有必要判断b吗?
\end{description}
\end{block}

\end{frame}

\begin{frame}[fragile]{逻辑运算符小练习}


\begin{block}{}
\begin{Verbatim}[numbers=left,frame=single,rulecolor=\color{red}]
let n = 4:
    (2 < n) or (n < 0)
    (n == 4) and (n > 5)
    (2 < n) and (n < 0)
    not (n >2)
\end{Verbatim}
\end{block}

\end{frame}

\section{判断结构}
\begin{frame}[fragile]{判断结构}
判断结构(也称为分支结构)允许程序根据特定条件的真假来决定执行哪些语句。
\begin{block}{if的形式\footnote{if可以嵌套,此处没有展示。}}
\begin{columns}
\column{2.8cm}{if}
\begin{Verbatim}[numbers=none,frame=single,rulecolor=\color{red}]
if condition:
    code…
\end{Verbatim}
\column{2.8cm}{if-else}
\begin{Verbatim}[numbers=none,frame=single,rulecolor=\color{red}]
if condition:
    code…
else:
    code…
\end{Verbatim}
\column{3.1cm}{if-elif}
\begin{Verbatim}[numbers=none,frame=single,rulecolor=\color{red}]
if condition:
    code…
elif condition:
    code…
\end{Verbatim}
\column{3.1cm}{if-elif-else}
\begin{Verbatim}[numbers=none,frame=single,rulecolor=\color{red}]
if condition:
    code…
elif condition:
    code…
else:
    code…
\end{Verbatim}
\end{columns}

\end{block}
\end{frame}
\begin{frame}[fragile]{if 示例代码}
\begin{Verbatim}[numbers=none,frame=single,rulecolor=\color{red}]
a,b = 2,3
\end{Verbatim}
\begin{columns}
\column{3.5cm}{if}
\begin{Verbatim}[numbers=none,frame=single,rulecolor=\color{red}]
if a<b:
    print("a<b")
    \end{Verbatim}
\column{3.5cm}{if-else}
\begin{Verbatim}[numbers=none,frame=single,rulecolor=\color{red}]
if a<b:
    print("a<b")
else:
    print("a>=b")
\end{Verbatim}
\column{3.5cm}{if-elif-else}
\begin{Verbatim}[numbers=none,frame=single,rulecolor=\color{red}]
if a<b:
    print("a<b")
elif a>b:
    print("a>b")
else:
    print("a=b")
\end{Verbatim}
\end{columns}
\end{frame}
\begin{frame}[fragile]{if练习题}
从终端输入两个字符,判断其是否为数值。
\begin{block}{Tips}
\begin{Verbatim}[numbers=left,frame=single,rulecolor=\color{red}]
>>> help(str.isdigit)
isdigit(self, /)
    Return True if the string is a digit string, 
    False otherwise.
\end{Verbatim}
\end{block}
\end{frame}
\begin{frame}[fragile]{if练习题答案\footnote{这里用到了str模块中的方法,为何没有在程序起始位置import str?}}
\textit{Q:从终端输入两个字符,判断其是否为数值。}
\begin{Verbatim}[numbers=left,frame=single,rulecolor=\color{red}]
a=input("input 1st:")
b=input("input 2nd:")
if a.isdigit() and b.isdigit():
    print("Both of them is digit.")
elif a.isdigit():
    print("b is not a digit.")
elif b.isdigit():
    print("a is not a digit.")
else:
    print("Both of them is not digit.")
\end{Verbatim}

\end{frame}

\section{循环结构}
\begin{frame}[fragile]{循环结构}
循环也是程序设计中最重要的结构之一
,是程序中可以重复执行的一段代码。有两种循环形式:
\begin{itemize}
\item while
\item for
\end{itemize}


\end{frame}
\subsection{while循环}

\begin{frame}[fragile]{while}
\begin{block}{语法}
\begin{Verbatim}[numbers=left,frame=single,rulecolor=\color{red}]
while condition:
    code…
\end{Verbatim}
\end{block}
第1行代码称为循环头,\textit{code…}部分称为循环体。

当碰到\textit{while}代码时,首先判断\textit{condition}是否成立,若成立,则执行循环体,每执行完一次循环体后都会再次判断\textit{condition}是否成立。
\end{frame}
\begin{frame}[fragile]{while小练习}
使用while循环,求出式子\begin{equation*}
2+4+6+…+100
\end{equation*}的值。
\end{frame}
\subsection{for循环}
\begin{frame}[fragile]{for}
\begin{block}{特点}
\textit{python}中的\textit{for}循环用来迭代(遍历)一系列的值,比如等差数列、字符串、列表、元组等序列中的元素。
\end{block}
\begin{block}{语法}
\begin{Verbatim}[numbers=left,frame=single,rulecolor=\color{red}]
for var in sequence:
    code…
\end{Verbatim}
\end{block}
\end{frame}

\begin{frame}[fragile]{使用\textit{for}遍历—字符串、列表、元组}

\begin{columns}
\column{3.2cm}{字符串}
\begin{Verbatim}[numbers=none,frame=single,rulecolor=\color{red}]
s = 'gzu'
for e in s:
    print(e)
g
z
u  
\end{Verbatim}

\column{4cm}{列表}
\begin{Verbatim}[numbers=none,frame=single,rulecolor=\color{red}]
mList = ['g','u',5]
for e in mList:
    print(e)
g
u
5
\end{Verbatim}
\column{4.2cm}{元组}
\begin{Verbatim}[numbers=none,frame=single,rulecolor=\color{red}]
mTuple = ('g','z',5)
for e in mTuple:
    print(e)
g
z
5
\end{Verbatim}
\end{columns}
\end{frame}

\begin{frame}[fragile]{使用\textit{for}遍历—等差数列}
与\textit{c、java}等语言不同,\textit{python}里的\textit{for}循环,在处理类似下列代码所示情形时,需要通过遍历一个等差数列来实现。
\begin{Verbatim}[numbers=left,frame=single,rulecolor=\color{red}]
int i, sum=0;
for(i=0; i<100; i++){
    sum += i;
}
\end{Verbatim}
\end{frame}

\begin{frame}[fragile]{如何产生等差数列}
在\textit{python}中,可以使用\textit{range()}函数\footnote{range类的构造方法}来产生等差数列,从而供\textit{for}循环使用,控制循环体。
\begin{Verbatim}[numbers=left,frame=single,rulecolor=\color{red}]
>>> help(range)
Help on class range in module builtins:
class range(object)
   range(stop) -> range object
   range(start, stop[, step]) -> range object
   
Return an object that produces a sequence of integers
 from start (inclusive) to stop (exclusive) by step.  
Start defaults to 0, and stop is omitted!  
range(4) produces 0, 1, 2, 3.
When step is given, 
it specifies the increment (or decrement).
\end{Verbatim}
\end{frame}

\begin{frame}[fragile]{使用\textit{for}遍历—等差数列\textit{range()}}

\begin{columns}
\column{3.2cm}{\textit{range(stop)}}
\begin{Verbatim}[numbers=none,frame=single,rulecolor=\color{red}]
l = range(3)
for e in l:
    print(e)
0
1
2
\end{Verbatim}

\column{4cm}{\textit{range(start, stop)}}
\begin{Verbatim}[numbers=none,frame=single,rulecolor=\color{red}]
ll = range(2,5)
for e in ll:
    print(e)
2
3
4
\end{Verbatim}
\column{4.2cm}{\textit{range(start, stop, step)}}
\begin{Verbatim}[numbers=none,frame=single,rulecolor=\color{red}]
lll = range(2,10,5)
for e in lll:
    print(e)
2
7
\end{Verbatim}
\end{columns}
\end{frame}

\begin{frame}[fragile]{for小练习}
使用for循环,求出式子\begin{equation*}
1+3+5+…+99
\end{equation*}的值。
\end{frame}
\subsection{跳出循环和无限循环}

\begin{frame}{\textit{break、continue}\footnote{通常出现在if语句里面。}}
\begin{columns}
\column{4cm}{\textit{break}}

在循环体中使用\textit{break}关键字,会使程序\textcolor{red}{停止执行循环体},并跳转到循环体外,执行后续代码。
\column{4cm}{\textit{continue}}

在循环体中使用\textit{continue}关键字,\textcolor{red}{会终止本轮循环},程序跳转到循环的头部。
\end{columns}
\end{frame}
\begin{frame}[fragile]{无限循环}
某些特殊的情况下,会使用到无限循环。在使用无限循环的时候,要注意在适当的情况下,结束循环。
\begin{Verbatim}[numbers=left,frame=single,rulecolor=\color{red}]
i = 0
while True:
    i += 1
    if i > 100:
        print("termination.")
        break
\end{Verbatim}
\end{frame}

%----------------------------------------------------------------------------------------
%	PRESENTATION SLIDES
%----------------------------------------------------------------------------------------
\section{homework}
\begin{frame}{Homework}
\begin{enumerate}
\item
求出前10个素数\footnote{质数(prime number)又称素数,有无限个。质数定义为在大于1的自然数中,除了1和它本身以外不再有其他因数。}。
\item
使用辗转相除法\footnote{用较大数除以较小数,再用出现的余数(第一余数)去除除数,再用出现的余数(第二余数)去除第一余数,如此反复,直到最后余数是0为止。如果是求两个数的最大公约数,那么最后的除数就是这两个数的最大公约数。}求任意两个正整数的最大公约数。
%\item
%找零钱。输入收费金额和客户所付的钱,给出需要怎么找零。要求找零的钱币张数最少。(即需要多少1元的,多少2元的,多少5元的等)
\end{enumerate}
\end{frame}
\section{Q\&A}
\begin{frame}
\center{\Huge{Q\&A}}
\end{frame}


%How do I uninstall?
%
%	1.	Remove /Applications/Wireshark.app
%	2.	Remove /Library/Application Support/Wireshark
%	3.	Remove the wrapper scripts from /usr/local/bin
%	4.	Unload the org.wireshark.ChmodBPF.plist launchd job
%	5.	Remove /Library/LaunchDaemons/org.wireshark.ChmodBPF.plist
%	6.	Remove the access_bpf group.
%	7.	Remove /etc/paths.d/Wireshark
%	8.	Remove /etc/manpaths.d/Wireshark
\end{document} 

