% !TEX encoding = UTF-8 Unicode
%%%%%%%%%%%%%%%%%%%%%%%%%%%%%%%%%%%%%%%%%
% Beamer Presentation
% LaTeX Template
% Version 1.0 (10/11/12)
%
% This template has been downloaded from:
% http://www.LaTeXTemplates.com
%
% License:
% CC BY-NC-SA 3.0 (http://creativecommons.org/licenses/by-nc-sa/3.0/)
%
%%%%%%%%%%%%%%%%%%%%%%%%%%%%%%%%%%%%%%%%%

%----------------------------------------------------------------------------------------
%	PACKAGES AND THEMES
%----------------------------------------------------------------------------------------

\documentclass{beamer}

\mode<presentation> {

% The Beamer class comes with a number of default slide themes
% which change the colors and layouts of slides. Below this is a list
% of all the themes, uncomment each in turn to see what they look like.

%\usetheme{default}
%\usetheme{AnnArbor}
%\usetheme{Antibes}
%\usetheme{Bergen}
%\usetheme{Berkeley}
%\usetheme{Berlin}
%\usetheme{Boadilla}
%\usetheme{CambridgeUS}
%\usetheme{Copenhagen}
%\usetheme{Darmstadt}
%\usetheme{Dresden}
%\usetheme{Frankfurt}
%\usetheme{Goettingen}
%\usetheme{Hannover}
%\usetheme{Ilmenau}
%\usetheme{JuanLesPins}
%\usetheme{Luebeck}
\usetheme{Madrid}
%\usetheme{Malmoe}
%\usetheme{Marburg}
%\usetheme{Montpellier}
%\usetheme{PaloAlto}
%\usetheme{Pittsburgh}
%\usetheme{Rochester}
%\usetheme{Singapore}
%\usetheme{Szeged}
%\usetheme{Warsaw}

% As well as themes, the Beamer class has a number of color themes
% for any slide theme. Uncomment each of these in turn to see how it
% changes the colors of your current slide theme.

%\usecolortheme{albatross}
%\usecolortheme{beaver}
%\usecolortheme{beetle}
%\usecolortheme{crane}
%\usecolortheme{dolphin}
%\usecolortheme{dove}
%\usecolortheme{fly}
%\usecolortheme{lily}
%\usecolortheme{orchid}
%\usecolortheme{rose}
%\usecolortheme{seagull}
%\usecolortheme{seahorse}
%\usecolortheme{whale}
%\usecolortheme{wolverine}

%\setbeamertemplate{footline} % To remove the footer line in all slides uncomment this line
%\setbeamertemplate{footline}[page number] % To replace the footer line in all slides with a simple slide count uncomment this line

%\setbeamertemplate{navigation symbols}{} % To remove the navigation symbols from the bottom of all slides uncomment this line
}

\usepackage{graphicx} % Allows including images
\usepackage{booktabs} % Allows the use of \toprule, \midrule and \bottomrule in tables
\usepackage{xeCJK}
\usepackage{color}

%----------------------------------------------------------------------------------------
%	TITLE PAGE
%----------------------------------------------------------------------------------------

\title[Linux System Programming]{Linux System Programming} % The short title appears at the bottom of every slide, the full title is only on the title page

\author{Zhang Haining} % Your name
\institute[cs.gzu] % Your institution as it will appear on the bottom of every slide, may be shorthand to save space
{
Guizhou University \\ % Your institution for the title page
\medskip
\textit{hnzhang1@gzu.edu.cn} % Your email address
}
\date{\today} % Date, can be changed to a custom date

\begin{document}

\begin{frame}
\titlepage % Print the title page as the first slide
\end{frame}

\begin{frame}
\frametitle{Overview} % Table of contents slide, comment this block out to remove it
\tableofcontents % Throughout your presentation, if you choose to use \section{} and \subsection{} commands, these will automatically be printed on this slide as an overview of your presentation
\end{frame}

%----------------------------------------------------------------------------------------
%	PRESENTATION SLIDES
%----------------------------------------------------------------------------------------

%------------------------------------------------
\section{System Programming} % Sections can be created in order to organize your presentation into discrete blocks, all sections and subsections are automatically printed in the table of contents as an overview of the talk
%------------------------------------------------
\begin{frame}
\frametitle{System Programming}
System programming is the art of writing system software.System software lives at a low level, interfacing directly with the \textcolor{red}{kernel and core system libraries}.

There are three cornerstones to system programming in Linux: 
\begin{enumerate}
\item system calls
\item the C library
\item the C compiler
\end{enumerate} 
\end{frame}
%\subsection{System Calls} % A subsection can be created just before a set of slides with a common theme to further break down your presentation into chunks

\begin{frame}
\frametitle{System Calls}
System programming starts with system calls.System calls (often shorted to syscalls) are \textcolor{red}{function invocations} made from \textcolor{red}{user space}—your text editor, favorite game, and so on—\textcolor{red}{into the kernel} (the core internals of the system) in order to request some service or resource from the operating system. 
\end{frame}
\begin{frame}
\frametitle{Invoke system calls}
It is not possible to directly link user-space applications with kernel space. For reasons of security and reliability, user-space applications must not be allowed to directly execute kernel code or manipulate kernel data. Instead, the kernel must provide a mechanism by which a user-space application can "signal" the kernel that it wishes to invoke a system call. The application can then trap into the kernel through this well-defined mechanism, and execute only code that the kernel allows it to execute. The exact mechanism varies from architecture to architecture.
\end{frame}


%------------------------------------------------
\begin{frame}
\frametitle{The C Library}
The C library (\textcolor{red}{libc}) is at the heart of Unix applications. Even when you're programming in another language, the C library is most likely in play, wrapped by the higher-level libraries, providing core services, and facilitating system call invocation. On modern Linux systems, the C library is provided by GNU libc, abbreviated \textcolor{red}{glibc}, and pronounced gee-lib-see or, less commonly, glib-see.

The GNU C library provides more than its name suggests. In addition to implementing the standard C library, glibc provides wrappers for system calls, threading support, and basic application facilities.
\end{frame}

%-------------------------------------------------
\begin{frame}
\frametitle{The C Compiler}
In Linux, the standard C compiler is provided by the \textcolor{red}{GNU Compiler Collection (gcc)}. Originally, gcc was GNU's version of cc, the C Compiler. Thus, gcc stood for \textcolor{red}{GNU C Compiler}. Over time, support was added for more and more languages. Consequently, nowadays gcc is used as the generic name for the family of GNU compilers. However, gcc is also the binary used to invoke the C compiler. In this course, when we talk of gcc, we typically mean the program gcc, unless context suggests otherwise.

\end{frame}

%------------------------------------------------

\begin{frame}
\frametitle{用户程序vs库函数vs系统调用}

\begin{figure}
\includegraphics[width=1\linewidth]{601}
\end{figure}
\end{frame}
%------------------------------------------------
\section{GCC}
\begin{frame}
\frametitle{GCC}
The GNU Compiler Collection includes front ends for C, C++, Objective-C, Fortran, Ada, and Go, as well as libraries for these languages (libstdc++,...). GCC was originally written as the compiler for the GNU operating system. The GNU system was developed to be 100\% free software, free in the sense that it respects the user's freedom.

GCC, formerly for "GNU C Compiler", has grown over times to support many languages such as C (gcc), C++ (g++), Objective-C, Objective-C++, Java (gcj), Fortran (gfortran), Ada (gnat), Go (gccgo), OpenMP, Cilk Plus, and OpenAcc. It is now referred to as "GNU Compiler Collection". The mother site for GCC is http://gcc.gnu.org/. The current version is GCC 7.3, released on 2018-01-25.
\end{frame}
%---------------------------------------------
\begin{frame}
\frametitle{GNU Toolchain}
GCC is a key component of "GNU Toolchain", for developing applications, as well as operating systems. The GNU Toolchain includes:
\begin{itemize}
\item GNU Compiler Collection (GCC): a compiler suit that supports many languages, such as C/C++ and Objective-C/C++
\item GNU Make: an automation tool for compiling and building applications
\item GNU Binutils: a suit of binary utility tools, including linker and assembler
\item GNU Debugger (GDB)
\item GNU Autotools: A build system including Autoconf, Autoheader, Automake and Libtool
\item GNU Bison: a parser generator (similar to lex and yacc)
\end{itemize}
\end{frame}
\begin{frame}
\frametitle{Platform}
GCC is portable and run in many operating platforms. GCC (and GNU Toolchain) is currently available on all Unixes. They are also ported to Windows (by MinGW and Cygwin). GCC is also a cross-compiler, for producing executables on different platform.
\end{frame}
\begin{frame}
\frametitle{Install gcc}
\begin{itemize}
\item Unix/Linux/Mac GCC(GNU Toolchain) is included.
\item Windows
\begin{itemize}
\item Cygwin GCC: Cygwin is a Unix-like environment and command-line interface for Microsoft Windows. Cygwin is huge and includes most of the Unix tools and utilities. It also included the commonly-used Bash shell.
\item MinGW: MinGW (Minimalist GNU for Windows) is a port of the GNU Compiler Collection (GCC) and GNU Binutils for use in Windows. It also included MSYS (Minimal System), which is basically a Bourne shell (bash).
\end{itemize}
\end{itemize}
\end{frame}

\begin{frame}[fragile]
\frametitle{Versions and Help}
We can display the version of GCC with --version/-v option, and get help with --help option:
\begin{example}[show version and help of GCC]
\begin{verbatim}
$ gcc -v
$ gcc --version
$ gcc --help
\end{verbatim}
\end{example}
\end{frame}

\subsection{Using GCC}
\begin{frame}
\frametitle{编写c程序的流程}
\begin{figure}
\includegraphics[width=1\linewidth]{602}
\end{figure}
\end{frame}
\begin{frame}
\frametitle{编写c程序的流程}
\begin{figure}
\includegraphics[width=1\linewidth]{603}
\end{figure}
\end{frame}

\begin{frame}[fragile]
\frametitle{hello.c}
\begin{example}
\begin{verbatim}
// hello.c
#include <stdio.h>

int main(){
        printf("Hello, C.\n");
        return 0;
}
\end{verbatim}
\end{example}
\end{frame}

%------------------------------------------
\begin{frame}
\frametitle{Blocks of Highlighted Text}
\begin{block}{Block 1}
Lorem ipsum dolor sit amet, consectetur adipiscing elit. Integer lectus nisl, ultricies in feugiat rutrum, porttitor sit amet augue. Aliquam ut tortor mauris. Sed volutpat ante purus, quis accumsan dolor.
\end{block}

\begin{block}{Block 2}
Pellentesque sed tellus purus. Class aptent taciti sociosqu ad litora torquent per conubia nostra, per inceptos himenaeos. Vestibulum quis magna at risus dictum tempor eu vitae velit.
\end{block}

\begin{block}{Block 3}
Suspendisse tincidunt sagittis gravida. Curabitur condimentum, enim sed venenatis rutrum, ipsum neque consectetur orci, sed blandit justo nisi ac lacus.
\end{block}
\end{frame}

%------------------------------------------------

\begin{frame}[fragile]
\frametitle{Multiple Columns}
\begin{columns}[c] % The "c" option specifies centered vertical alignment while the "t" option is used for top vertical alignment

\column{.45\textwidth} % Left column and width
\textbf{Heading}
\begin{example}[hello.c]
\begin{verbatim}
// hello.c
#include <stdio.h>

int main(){
        printf("Hello, C.\n");
        return 0;
}
\end{verbatim}
\end{example}
\column{.5\textwidth} % Right column and width
Lorem ipsum dolor sit amet, consectetur adipiscing elit. Integer lectus nisl, ultricies in feugiat rutrum, porttitor sit amet augue. Aliquam ut tortor mauris. Sed volutpat ante purus, quis accumsan dolor.

\end{columns}
\end{frame}

%------------------------------------------------
\section{Second Section}
%------------------------------------------------

\begin{frame}
\frametitle{Table}
\begin{table}
\begin{tabular}{l l l}
\toprule
\textbf{Treatments} & \textbf{Response 1} & \textbf{Response 2}\\
\midrule
Treatment 1 & 0.0003262 & 0.562 \\
Treatment 2 & 0.0015681 & 0.910 \\
Treatment 3 & 0.0009271 & 0.296 \\
\bottomrule
\end{tabular}
\caption{Table caption}
\end{table}
\end{frame}

%------------------------------------------------

\begin{frame}
\frametitle{Theorem}
\begin{theorem}[Mass--energy equivalence]
$E = mc^2$
\end{theorem}
\end{frame}

%------------------------------------------------

\begin{frame}[fragile] % Need to use the fragile option when verbatim is used in the slide
\frametitle{Verbatim}
\begin{example}[Theorem Slide Code]
\begin{verbatim}
\begin{frame}
\frametitle{Theorem}
\begin{theorem}[Mass--energy equivalence]
$E = mc^2$
\end{theorem}
\end{frame}\end{verbatim}
\end{example}
\end{frame}

%------------------------------------------------

\begin{frame}
\frametitle{Figure}
Uncomment the code on this slide to include your own image from the same directory as the template .TeX file.
\begin{figure}
\includegraphics[width=0.8\linewidth]{601}
\end{figure}
\end{frame}

%------------------------------------------------

\begin{frame}[fragile] % Need to use the fragile option when verbatim is used in the slide
\frametitle{Citation}
An example of the \verb|\cite| command to cite within the presentation:\\~

This statement requires citation \cite{p1}.
\end{frame}

%------------------------------------------------

\begin{frame}
\frametitle{References}
\footnotesize{
\begin{thebibliography}{99} % Beamer does not support BibTeX so references must be inserted manually as below
\bibitem[Smith, 2012]{p1} John Smith (2012)
\newblock Title of the publication
\newblock \emph{Journal Name} 12(3), 45 -- 678.
\end{thebibliography}
}
\end{frame}

%------------------------------------------------

\begin{frame}
\Huge{\centerline{The End}}
\end{frame}

%----------------------------------------------------------------------------------------

\end{document} 