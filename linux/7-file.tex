% !TEX encoding = UTF-8 Unicode
%%%%%%%%%%%%%%%%%%%%%%%%%%%%%%%%%%%%%%%%%
% Beamer Presentation
% LaTeX Template
% Version 1.0 (10/11/12)
%
% This template has been downloaded from:
% http://www.LaTeXTemplates.com
%
% License:
% CC BY-NC-SA 3.0 (http://creativecommons.org/licenses/by-nc-sa/3.0/)
%
%%%%%%%%%%%%%%%%%%%%%%%%%%%%%%%%%%%%%%%%%

%----------------------------------------------------------------------------------------
%	PACKAGES AND THEMES
%----------------------------------------------------------------------------------------

\documentclass{beamer}

\mode<presentation> {

% The Beamer class comes with a number of default slide themes
% which change the colors and layouts of slides. Below this is a list
% of all the themes, uncomment each in turn to see what they look like.

%\usetheme{default}
%\usetheme{AnnArbor}
%\usetheme{Antibes}
%\usetheme{Bergen}
%\usetheme{Berkeley}
%\usetheme{Berlin}
%\usetheme{Boadilla}
%\usetheme{CambridgeUS}
%\usetheme{Copenhagen}
%\usetheme{Darmstadt}
%\usetheme{Dresden}
%\usetheme{Frankfurt}
%\usetheme{Goettingen}
%\usetheme{Hannover}
%\usetheme{Ilmenau}
%\usetheme{JuanLesPins}
%\usetheme{Luebeck}
\usetheme{Madrid}
%\usetheme{Malmoe}
%\usetheme{Marburg}
%\usetheme{Montpellier}
%\usetheme{PaloAlto}
%\usetheme{Pittsburgh}
%\usetheme{Rochester}
%\usetheme{Singapore}
%\usetheme{Szeged}
%\usetheme{Warsaw}

% As well as themes, the Beamer class has a number of color themes
% for any slide theme. Uncomment each of these in turn to see how it
% changes the colors of your current slide theme.

%\usecolortheme{albatross}
%\usecolortheme{beaver}
%\usecolortheme{beetle}
%\usecolortheme{crane}
%\usecolortheme{dolphin}
%\usecolortheme{dove}
%\usecolortheme{fly}
%\usecolortheme{lily}
%\usecolortheme{orchid}
%\usecolortheme{rose}
%\usecolortheme{seagull}
%\usecolortheme{seahorse}
%\usecolortheme{whale}
%\usecolortheme{wolverine}

%\setbeamertemplate{footline} % To remove the footer line in all slides uncomment this line
%\setbeamertemplate{footline}[page number] % To replace the footer line in all slides with a simple slide count uncomment this line

%\setbeamertemplate{navigation symbols}{} % To remove the navigation symbols from the bottom of all slides uncomment this line
}

\usepackage{graphicx} % Allows including images
\usepackage{booktabs} % Allows the use of \toprule, \midrule and \bottomrule in tables
\usepackage{xeCJK}
\usepackage{color}
\usepackage{listings}
\usepackage{tikz}


%----------------------------------------------------------------------------------------
%	TITLE PAGE
%----------------------------------------------------------------------------------------

\title[File]{File} % The short title appears at the bottom of every slide, the full title is only on the title page

\author{张海宁} % Your name
\institute[贵大计算机] % Your institution as it will appear on the bottom of every slide, may be shorthand to save space
{
贵州大学 \\ % Your institution for the title page
\medskip
\textit{hnzhang1@gzu.edu.cn} % Your email address
}
\date{\today} % Date, can be changed to a custom date

\begin{document}

\begin{frame}
\titlepage % Print the title page as the first slide
\end{frame}

\begin{frame}
\frametitle{Overview} % Table of contents slide, comment this block out to remove it
\tableofcontents % Throughout your presentation, if you choose to use \section{} and \subsection{} commands, these will automatically be printed on this slide as an overview of your presentation
\end{frame}

%----------------------------------------------------------------------------------------
%	PRESENTATION SLIDES
%----------------------------------------------------------------------------------------

%------------------------------------------------
\section{文件结构} % Sections can be created in order to organize your presentation into discrete blocks, all sections and subsections are automatically printed in the table of contents as an overview of the talk
%------------------------------------------------
\begin{frame}
\frametitle{文件结构}
Linux中,一切(或几乎一切)都是文件。

这意味着针对串口,打印机等设备的操作可以像文件一样来进行操作。

文件可以分为以下两类:
\begin{enumerate}
\item
目录

特殊的文件
\item
文件和设备
\end{enumerate}
Linux系统是通过inode来识别文件的,文件名只是为了用户的使用方便。
\end{frame}
\begin{frame}
\frametitle{inode}
inode是linux系统中的一种数据结构。它存储了文件系统对象(包括文件、目录、设备文件、socket、管道, 等等)的元信息数据,但不包括数据内容或者文件名。

文件系统创建(格式化)时,就把存储区域分为两大连续的存储区域。一个用来保存文件系统对象的元信息数据,这是由inode组成的表,每个inode默认是256字节或者128字节。另一个用来保存“文件系统对象”的内容数据,划分为512字节的扇区,以及由8个扇区组成的4K字节的块。块是读写时的基本单位。一个文件系统的inode的总数是固定的。这限制了该文件系统所能存储的文件系统对象的总数目。典型的实现下,所有inode占用了文件系统1%左右的存储容量。

文件系统中每个“文件系统对象”对应一个“inode”数据,并用一个整数值来辨识。这个整数常被称为inode号码(“i-number”或“inode number”)。由于文件系统的inode表的存储位置、总条目数量都是固定的,因此可以用inode号码去索引查找inode表。

Inode存储了文件系统对象的一些元信息,如所有者、访问权限(读、写、执行)、类型(是文件还是目录)、内容修改时间、inode修改时间、上次访问时间、对应的文件系统存储块的地址,等等。知道了1个文件的inode号码,就可以在inode元数据中查出文件内容数据的存储地址。

文件名与目录名是“文件系统对象”便于使用的别名。一个文件系统对象可以有多个别名,但只能有一个inode,并用这个inode来索引文件系统对象的存储位置。


inode不包含文件名或目录名的字符串,只包含文件或目录的“元信息”。
Unix的文件系统的目录也是一种文件。打开目录,实际上就是读取“目录文件”。目录文件的结构是一系列目录项(dirent)的列表。每个目录项,由两部分组成:所包含文件或目录的名字,以及该文件或目录名对应的inode号码。
文件系统中的一个文件是指存放在其所属目录的“目录文件”中的一个目录项,其所对应的inode的类别为“文件”;文件系统中的一个目录是指存放在其“父目录文件”中的一个目录项,其所对应的inode的类别为“目录”。可见,多个“文件”可以对应同一个inode;多个“目录”可以对应同一个inode。
文件系统中如果两个文件或者两个目录具有相同的inode号码,那么就称它们是“硬链接”关系。实际上都是这个inode的别名。换句话说,一个inode对应的所有文件(或目录)中的每一个,都对应着文件系统某个“目录文件”中唯一的一个目录项。
创建一个目录时,实际做了3件事:在其“父目录文件”中增加一个条目;分配一个inode;再分配一个存储块,用来保存当前被创建目录包含的文件与子目录。被创建的“目录文件”中自动生成两个子目录的条目,名称分别是:“.”和“..”。前者与该目录具有相同的inode号码,因此是该目录的一个“硬链接”。后者的inode号码就是该目录的父目录的inode号码。所以,任何一个目录的"硬链接"总数,总是等于它的子目录总数(含隐藏目录)加2。即每个“子目录文件”中的“..”条目,加上它自身的“目录文件”中的“.”条目,再加上“父目录文件”中的对应该目录的条目。
通过文件名打开文件,实际上是分成三步实现:首先,操作系统找到这个文件名对应的inode号码;其次,通过inode号码,获取inode信息;最后,根据inode信息,找到文件数据所在的block,读出数据。
Linux系统使用struct inode作为数据结构名称。BSD派生的系统,使用vnode名称,其中v表示“virtual file system”。


OSIX inode
POSIX标准强制规范了文件系统的行为。每个“文件系统对象”必须具有:

以字节为单位表示的文件大小。
设备ID,标识容纳该文件的设备。
文件所有者的User ID。
文件的Group ID
文件的模式(mode),确定了文件的类型,以及它的所有者、它的group、其它用户访问此文件的权限。
额外的系统与用户标志(flag),用来保护该文件。
3个时间戳,记录了inode自身被修改(ctime, inode change time)、文件内容被修改(mtime, modification time)、最后一次访问(atime, access time)的时间。
1个链接数,表示有多少个硬链接指向此inode。
到文件系统存储位置的指针。通常是1K字节或者2K字节的存储容量为基本单位。
使用stat系统调用可以查询一个文件的inode号码及一些元信息。
\end{frame}
%\subsection{System Calls} % A subsection can be created just before a set of slides with a common theme to further break down your presentation into chunks

\begin{frame}
\frametitle{目录}
\begin{tikzpicture}
\node(/) at (7,10) {/};
\node(bin) at (5,8) {bin};
\node(home) at (7,8) {home};
\node(dev) at (9,8) {dev};
\draw (bin) -- (/) -- (dev);
\draw (/) -- (home);
\node (zhn) at (5,6) {zhn};
\node (wys) at (7,6) {wys};
\node (students) at (9,6) {students};
\draw (home) -- (zhn);  
\draw (home) -- (wys); 
\draw (home) -- (students); 
%\node (public\_html) at (3,4) {public\_html};
\node (linux) at (2,4) {linux};
\node (javaweb) at (7,4) {javaweb};
%\draw (zhn) -- (public\_html);
\draw (zhn) -- (linux);
\draw (zhn) -- (javaweb);


\end{tikzpicture}
\end{frame}
\begin{frame}
\frametitle{Invoke system calls}
It is not possible to directly link user-space applications with kernel space. For reasons of security and reliability, user-space applications must not be allowed to directly execute kernel code or manipulate kernel data. Instead, the kernel must provide a mechanism by which a user-space application can "signal" the kernel that it wishes to invoke a system call. The application can then trap into the kernel through this well-defined mechanism, and execute only code that the kernel allows it to execute. The exact mechanism varies from architecture to architecture.
\end{frame}


%------------------------------------------------
\begin{frame}
\frametitle{The C Library}
The C library (\textcolor{red}{libc}) is at the heart of Unix applications. Even when you're programming in another language, the C library is most likely in play, wrapped by the higher-level libraries, providing core services, and facilitating system call invocation. On modern Linux systems, the C library is provided by GNU libc, abbreviated \textcolor{red}{glibc}, and pronounced gee-lib-see or, less commonly, glib-see.

The GNU C library provides more than its name suggests. In addition to implementing the standard C library, glibc provides wrappers for system calls, threading support, and basic application facilities.
\end{frame}

%-------------------------------------------------
\begin{frame}
\frametitle{The C Compiler}
In Linux, the standard C compiler is provided by the \textcolor{red}{GNU Compiler Collection (gcc)}. Originally, gcc was GNU's version of cc, the C Compiler. Thus, gcc stood for \textcolor{red}{GNU C Compiler}. Over time, support was added for more and more languages. Consequently, nowadays gcc is used as the generic name for the family of GNU compilers. However, gcc is also the binary used to invoke the C compiler. In this course, when we talk of gcc, we typically mean the program gcc, unless context suggests otherwise.

\end{frame}

%------------------------------------------------

\begin{frame}
\frametitle{用户程序vs库函数vs系统调用}

\begin{figure}
\includegraphics[width=1\linewidth]{601}
\end{figure}
\end{frame}
%------------------------------------------------
\section{GCC}
\begin{frame}
\frametitle{GCC}
The \textcolor{red}{GNU Compiler Collection} includes front ends for C, C++, Objective-C, Fortran, Ada, and Go, as well as libraries for these languages (libstdc++,...). GCC was originally written as the compiler for the GNU operating system. 

The GNU system was developed to be 100\% free software, free in the sense that it respects the \textcolor{red}{user's freedom}.

\end{frame}
\begin{frame}
\frametitle{The four essential freedoms}
A program is free software if the program's users have the four essential freedoms:
\begin{itemize}
\item
The freedom to \textcolor{red}{run} the program as you wish, for any purpose (freedom 0).
\item
The freedom to \textcolor{red}{study} how the program works, and \textcolor{red}{change} it so it does your computing as you wish (freedom 1). Access to the source code is a precondition for this.
\item
The freedom to \textcolor{red}{redistribute} copies so you can help others (freedom 2).
\item
The freedom to \textcolor{red}{distribute copies of your modified versions} to others (freedom 3). By doing this you can give the whole community a chance to benefit from your changes. Access to the source code is a precondition for this.
\end{itemize}
\url{https://www.gnu.org/philosophy/free-sw.html}

\end{frame}
%---------------------------------------------
\begin{frame}
\frametitle{GNU Toolchain}
GCC is a key component of "GNU Toolchain", for developing applications, as well as operating systems. The GNU Toolchain includes:
\begin{itemize}
\item GNU Compiler Collection (GCC): a compiler suit that supports many languages, such as C/C++ and Objective-C/C++
\item GNU Make: an automation tool for compiling and building applications
\item GNU Binutils: a suit of binary utility tools, including linker and assembler
\item GNU Debugger (GDB)
\item GNU Autotools: A build system including Autoconf, Autoheader, Automake and Libtool
\item GNU Bison: a parser generator (similar to lex and yacc)
\end{itemize}
\end{frame}
\begin{frame}
\frametitle{Platform}
GCC is portable and run in many operating platforms.

 GCC (and GNU Toolchain) is currently available on all \textcolor{red}{Unixes}. They are also ported to \textcolor{red}{Windows} (by MinGW and Cygwin). GCC is also a cross-compiler, for producing executables on different platform.
\end{frame}
\begin{frame}
\frametitle{Install gcc}
\begin{itemize}
\item Unix/Linux/Mac GCC(GNU Toolchain) is included.
\item Windows
\begin{itemize}
\item Cygwin GCC: Cygwin is a Unix-like environment and command-line interface for Microsoft Windows. Cygwin is huge and includes most of the Unix tools and utilities. It also included the commonly-used Bash shell.
\item MinGW: MinGW (Minimalist GNU for Windows) is a port of the GNU Compiler Collection (GCC) and GNU Binutils for use in Windows. It also included MSYS (Minimal System), which is basically a Bourne shell (bash).
\end{itemize}
\end{itemize}
\end{frame}

\begin{frame}[fragile]
\frametitle{Versions and Help}
We can display the version of GCC with --version/-v option, and get help with --help option:
\begin{example}[show version and help of GCC]
\begin{verbatim}
$ gcc -v
$ gcc --version
$ gcc --help
\end{verbatim}
\end{example}
\end{frame}

\subsection{Using GCC}
\begin{frame}
\frametitle{编写c程序的流程}
\begin{figure}
\includegraphics[width=1\linewidth]{602}
\end{figure}
\end{frame}
\begin{frame}
\frametitle{编写c程序的流程}
\begin{figure}
\includegraphics[width=1\linewidth]{603}
\end{figure}
\end{frame}

\begin{frame}[fragile]
\frametitle{hello.c}
\begin{example}
\begin{verbatim}
// hello.c
#include <stdio.h>

int main(){
        printf("Hello, C.\n");
        return 0;
}
\end{verbatim}
\end{example}
\end{frame}



%------------------------------------------
\begin{frame}
\frametitle{从.c文件到可执行文件}
\begin{block}{预处理}
gcc -E hello.c -o hello.i
\end{block}

\begin{block}{编译}
gcc -S hello.i

\end{block}

\begin{block}{汇编}
gcc -c hello.s -o hello.o

\end{block}
\begin{block}{链接}
gcc hello.o -o hello.out
\end{block}

\begin{block}{all in one}
gcc hello.c -o hello.out
\end{block}
\end{frame}

%-------------------------------------------------
\subsection{库文件}
\begin{frame}
\frametitle{静态库和动态库}
库文件是一组被预先编译的目标文件的集合,可以被链接到用户所编写的程序中。
\begin{itemize}
\item 静态库

通常以“.a”结尾(archive file),静态库的代码在编译时就会连接到用户的程序中
\item 动态库

通常以“.so”结尾(shared objects),用户的程序在运行时,按需要载入
\end{itemize}

\end{frame}
\subsection{补充工具}
\begin{frame}
\frametitle{几个有用的工具}
\begin{enumerate}
\item file

判断文件的类型
\item nm

显示目标文件的符号表,通常用于查看目标文件中是否定义了某个特定的函数

\item ldd

检查一个可执行文件并显示出其需要的动态库文件

\item
mtrace

内存溢出检测

\end{enumerate}
\end{frame}
%-----------------------------------------------
\section{make}
\begin{frame}
\frametitle{make}
依照makefile文件,make程序可以自动确定一个软件包的哪些部分需要重新编译。

makefile文件描述了建立可执行程序的一些规则。
\end{frame}
%------------------------------------------------

\begin{frame}[fragile]
\frametitle{makefile}
\begin{columns}[c] % The "c" option specifies centered vertical alignment while the "t" option is used for top vertical alignment

\column{.45\textwidth} % Left column and width
\textbf{hello.c}
\lstset{language=C}
\begin{lstlisting}

// hello.c
#include <stdio.h>

int main(){
    printf("Hello,C.\n");
    return 0;
}

\end{lstlisting}
\column{.5\textwidth} % Right column and width
\textbf{makefile}
\begin{lstlisting}
all:hello

hello:hello.o
        gcc -o hello hello.o

hello.o:hello.c
        gcc -c hello.c
clean:
        rm hello.o
\end{lstlisting}

\end{columns}
\end{frame}
\begin{frame}[fragile]
\frametitle{makefile的规则}
\begin{lstlisting}
target: pre-req-1 pre-req-2 ...
	command
\end{lstlisting}
The target and pre-requisites are separated by a colon (:). The command must be preceded by a tab (NOT spaces).
\end{frame}
\begin{frame}[fragile]
\frametitle{make和makefile}
\begin{lstlisting}
$ make
gcc -c hello.c
gcc -o hello hello.o
$ make clean
rm hello.o
$ ls
hello  hello.c  makefile
\end{lstlisting}
\end{frame}

\begin{frame}[fragile]
\frametitle{修改源文件与make的关系}
\begin{lstlisting}
$ make
gcc -c hello.c
gcc -o hello hello.o
$ make
make: Nothing to be done for 'all'.
$ vim hello.c
$ make
gcc -c hello.c
gcc -o hello hello.o
$ ./hello
Hello, C.
test make.
\end{lstlisting}
\end{frame}
%------------------------------------------------
\section{GDB}
%------------------------------------------------
\begin{frame}
\frametitle{GDB}
GDB(GNU Source-Level Debugger),是GNU的一个\textcolor{red}{命令行调试工具}。它主要可以完成以下功能:
\begin{itemize}
\item
设置运行环境和参数运行指定程序
\item
让程序在指定的条件下停止
\item
当程序停止时,检查发生了什么
\item
改变正在调试的程序,以修正某个bug,然后继续调试
\end{itemize}
\end{frame}
\begin{frame}
\frametitle{gdb部分命令}
\begin{table}
\begin{tabular}{l l l l }
\toprule
\textbf{命令} & \textbf{作用} & \textbf{命令} & \textbf{作用}\\
\midrule
file & 载入可执行文件 & list & 显示源代码 \\
run & 执行 & info local & 显示当前函数中变量值 \\
kill & 停止执行 & info break & 显示断点列表 \\
break & 设置断点 & info func & 显示所有函数名 \\
delete & 删除断点 & watch & 监视表达式的变化 \\
print & 显示表达式的值 & next & 执行下一条代码,但不进入 \\
shell命令 & 执行shell命令 & step & 执行下一条代码,进入函数 \\
\bottomrule
\end{tabular}
\caption{GDB部分命令}
\end{table}
\end{frame}

\begin{frame}[fragile]
\frametitle{gdb示例}
\begin{lstlisting}
$ vim gdb.c
$ cat gdb.c
#include <stdio.h>
int main(){
        int i=0;
        for(i=0;i<=15;i++){
                printf("the number is %d",i);
        }
}
$ gcc -ggdb -o gdbTest gdb.c
\end{lstlisting}
\end{frame}

\begin{frame}
\frametitle{gdb示例}
\includegraphics[width=0.8\linewidth]{605}

\end{frame}
%------------------------------------------------
\begin{frame}
\Huge{\centerline{总结}}
\begin{enumerate}
\item
System Call,C Compiler,C Library
\item
GCC
\item
make,makefile
\item
GDB
\end{enumerate}

\end{frame}

%------------------------------------------------
\begin{frame}
\frametitle{作业}
按以下要求进行编程练习:
\begin{enumerate}
\item
2个cpp文件:prog.cpp, aux.cpp
\item
1个h文件:aux.h
\item
aux.h头文件定义函数Max和Min,它们分别计算四个数(参数)的最大值和最小值;aux.cpp实现这两个函数
\item
prog.c中定义主函数,循环输入4个随机数,输出他们的最大和最小值
\item
编译该项目,调试、跟踪程序执行过程;并在控制台界面运行编译的可执行文件。
\item
\textcolor{red}{编写类似功能的c语言程序也可。}
\end{enumerate}
\end{frame}


%------------------------------------------------

\begin{frame}
\Huge{\centerline{The End}}
\end{frame}

%----------------------------------------------------------------------------------------

\end{document} 