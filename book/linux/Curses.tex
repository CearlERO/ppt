\chapter{Curses}
\section{目的}
熟悉linux中\textcolor{red}{Curses}库的使用,巩固\textcolor{red}{标准I/O库的读文件操作}。
\section{目标}
编写一个vimlike程序,要求至少能够读取并显示文件内容。
\section{实验过程}
\subsection{准备知识}

\subsubsection{相关的函数}
本次实验可能要用到的curses库函数有:
\begin{itemize}
\item
窗口相关

initscr, subwin, newwin, newpad, subpad, box, delwin, endwin
\item
字符串相关

getch, addch,  getstr, addstr
\end{itemize}
对于这些函数的具体用法,请使用\emph{man 3 xxx}来获取帮助。
\subsection{实现}
\subsubsection{构造主界面}
构造用户打开程序后的界面,代码如Figure \ref{Curses_mainui}所示,这段代码做了这些事情:
\begin{enumerate}
\item
构造了一个父窗口fwin
\item
为文件名fileName开辟了一块内存空间
\item
为本程序的所有窗口设置了背景色和前景色
\end{enumerate}
\begin{figure}
\begin{lstlisting}
#include<unistd.h>
#include<curses.h>
#include<stdlib.h>

void readAfile(char * fileName, WINDOW * win);
int main(){
WINDOW * fwin;
WINDOW * swin;
char * fileName;
fileName = (char *) malloc(32);
fwin=initscr();
//add some colors
if(has_colors()){
 start_color();
 init_pair(1,COLOR_GREEN,COLOR_BLACK);
 attron(COLOR_PAIR(1));
}
//give some tips
addstr("choose an option:\nO for open a file\nQ for quit");
\end{lstlisting}
\caption{构造主界面}
\label{Curses_mainui}
\end{figure}
\subsubsection{判断用户按键并作出相应的操作}
判断用户按键,代码如Figure \ref{Curses_keypress}所示,这段代码做了这些事情:
\begin{enumerate}
\item
如果用户输入了Q键,直接退出程序
\item
如果用户输入了O键,则打开一个子窗口,并接收用户输入
\item
打开用户在子窗口输入的文件名,并将文件内容显示到父窗口中
\end{enumerate}

\begin{figure}
\begin{lstlisting}
//detect the keypress
while(1){
 int key=getch();
 char key_press=(char)key;
 switch(key_press){
  case 'Q':
   endwin();
   exit(0);
  case 'O':
   swin = subwin(fwin,3,15,20,30);
   box(swin,'#','#');
   wmove(swin,1,2);
   wgetstr(swin,fileName);	
   wrefresh(swin);
   readAfile(fileName,fwin);
   delwin(swin);
 }
}
\end{lstlisting}
\caption{判断用户按键并执行相应的操作}
\label{Curses_keypress}
\end{figure}
\subsubsection{函数---读取指定的文件并显示到父窗口}
使用readAfile函数读取指定文件,并将文件内容显示到父窗口,代码如Figure \ref{Curses_readandview}所示。
\begin{figure}
\begin{lstlisting}
void readAfile(char * fileName,WINDOW * win){
	FILE * file = fopen(fileName,"r");
	int ch= fgetc(file);
	while(ch!=EOF){
		waddch(win,(char)ch);
		wrefresh(win);
		ch= fgetc(file);
	}	
	fclose(file);
}
\end{lstlisting}
\caption{读取文件并显示}
\label{Curses_readandview}
\end{figure}