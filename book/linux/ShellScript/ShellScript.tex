\chapter{ShellScript}
\section{目的}
\begin{enumerate}
\item
理解shell script的意义
\item
掌握shell script的写法
\end{enumerate}
\section{目标}
\begin{enumerate}
\item
写一脚本,计算你下一次的生日距离今天还有多少天。
\item
写一脚本,将服务器上所有的用户名显示出来。(按如下格式显示)
\begin{verbatim}
the 802 line is:1400170159

the 803 line is:1500170146

the 804 line is:1500170150
\end{verbatim}
\end{enumerate}
\section{实验过程}
\subsection{计算生日}
\subsubsection{分析}
本题目需要使用到date命令来获取当前的时间,并且需要输入生日的日期。
\subsubsection{难点}
日期差的计算。
\subsubsection{解决方法}
计算机中的时间表示是从1970-01-01 00:00:00开始计算,到目前为止经过了多长时间,可以借助这个知识来计算。
\subsubsection{代码}
此问题的脚本如Figure \ref{SS1}所示。
\begin{figure}
\begin{verbatim}
#!bin/bash
#calculate the day to next birthday
read -p"please input your birthday:(mmdd/0102)" birthday
year=`date +%Y`
yb=${year}${birthday}
now=`date +%Y%m%d`
nowSecond=`date +%s`
birthdaySecond=`date +%s -d "${yb}"`
bts=$((${birthdaySecond}/60/60/24))
nts=$((${nowSecond}/60/60/24))
echo "birthday to 19700101 is ${bts} days."
echo "today to 19700101 is ${nts} days."
if [ "${bts}" == "${nts}" ]; then
echo "Today is your birthday.Happy birthday to you!"
elif [ "${bts}" -gt "${nts}" ]; then
day=$((${bts}-${nts}))
echo "Your birthday is after ${day} days."
else
day=$((365-$((${nts}-${bts}))))
echo "Your birthday is after about ${day} days."
fi
\end{verbatim}
\caption{计算生日}
\label{SS1}
\end{figure}

\subsection{输出系统中的用户}
\subsubsection{分析}
要完成任务,需要知道当前系统中的用户名存储在哪个文件中,还要知道如何按要求输出。
\subsubsection{难点}
获取特定字段。
\subsubsection{解决方法}
可用cut命令截取要求的字段输出。(awk也可以。)
\subsubsection{代码}
此问题的脚本如Figure \ref{SS2}所示。
\begin{figure}
\begin{verbatim}
#!/bin/bash
#list all the user of this system.
i=1
for user in `cat /etc/passwd|cut -d: -f1`
do
echo "the ${i} line is:${user}"
i=$((i+1))
done\end{verbatim}
\caption{计算生日}
\label{SS2}
\end{figure}