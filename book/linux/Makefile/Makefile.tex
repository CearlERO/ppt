\chapter{Makefile}
\section{目的}
\begin{enumerate}
\item
熟悉make的使用
\item
熟悉makefile文件的写法
\end{enumerate}
\section{目标}
按以下要求进行编程练习:
\begin{enumerate}
\item
写2个cpp文件:prog.cpp, aux.cpp。
\item
写1个h文件:aux.h。
\item
aux.h头文件定义函数Max和Min,它们分别计算四个数(参数)的最大值和最小值;aux.cpp实现这两个函数。
\item
prog.c中定义主函数,循环输入4个随机数,输出他们的最大和最小值。
\item
编译该项目,调试、跟踪程序执行过程;并在控制台界面运行编译的可执行文件。
\item
编写类似功能的c语言程序也可。
\end{enumerate}
\section{实验过程}
\subsection{定义头文件}
头文件aux.h代码如Figure \ref{Mf1}所示。
\begin{figure}
\begin{verbatim}
#ifndef _AUX_H_
#define _AUX_H_

int Max(int a,int b,int c,int d);
int Min(int a,int b,int c,int d);

#endif
\end{verbatim}
\caption{定义头文件}
\label{Mf1}
\end{figure}

\subsection{实现头文件}
实现头文件的aux.c代码如Figure \ref{Mf2}所示。
\begin{figure}
\begin{verbatim}
#include <stdio.h>
#include "aux.h"

int Mx(int a, int b){
 if(a>b){
  return a;
 }else{
  return b;
 }
}
int Max(int a,int b,int c,int d){
 return Mx(Mx(a,b),Mx(c,d));
}
int Mn(int a, int b){
 if(a>b){
  return b;
 }else{
  return a;
 }
}
int Min(int a,int b,int c,int d){
 return Mn(Mn(a,b),Mn(c,d));
}
\end{verbatim}
\caption{实现头文件}
\label{Mf2}
\end{figure}


\subsection{编写主程序}
主程序prog.c代码如Figure \ref{Mf3}所示。
\begin{figure}
\begin{verbatim}
#include <stdio.h>
#include <stdlib.h>

#include "aux.h"

int main(){
 int i=0;
 for(i=0;i<4;i++){
  int a=rand();
  int b=rand();
  int c=rand();
  int d=rand(); 
  printf("the max number of %d,%d,%d,%d, is:%d\n",a,b,c,d,Max(a,b,c,d));    
  printf("the min number of %d,%d,%d,%d, is:%d\n",a,b,c,d,Min(a,b,c,d));
 }
}
\end{verbatim}
\caption{主程序}
\label{Mf3}
\end{figure}


\subsection{编写makefile文件}
一个简单的makefile文件如Figure \ref{Mf4}所示。
\begin{figure}
\begin{verbatim}
prog:prog.c aux.c
  gcc -o prog prog.c aux.c -I.
\end{verbatim}
\caption{simple makefile}
\label{Mf4}
\end{figure}
Figure \ref{Mf4}所示是最简单的makefile文件写法,但是存在一个问题,就是头文件编辑后,执行make不会重新编译。为解决这个问题,重新编写如Figure \ref{Mf5}所示的makefile文件。
\begin{figure}
\begin{verbatim}
CC=gcc
CFLAGS=-I.
DEPS=aux.h
OBJ=prog.o aux.o

%.o:%.c $(DEPS)
  $(CC) -c -o $@ $< $(CFLAGS)
prog:$(OBJ)
  $(CC) -o $@ $^ $(CFLAGS)
clean:
  rm *.o
\end{verbatim}
\caption{a smart makefile}
\label{Mf5}
\end{figure}

\textbf{参考资料}
\begin{enumerate}
\item
\url{http://www.gnu.org/software/make/manual/make.html}
\item
\url{http://www.cs.colby.edu/maxwell/courses/tutorials/maketutor/}
\end{enumerate}
